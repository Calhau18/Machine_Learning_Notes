\documentclass{article}
\title{Machine Learning Notes}
\date{2022}
\author{João Rocha}
	
\usepackage[utf8]{inputenc}
\usepackage{indentfirst}
\usepackage{comment}
\usepackage{amsmath}
\usepackage{amssymb}
\usepackage{mathtools, nccmath}
\usepackage{geometry}
\usepackage{graphicx}
 \geometry{
 a4paper,
 total={129mm,170mm},
 left=23mm,
 top=23mm,
 bottom=28mm,
 right=28mm,
 }

\newcommand{\R}{\mathbb{R}}
\newcommand{\N}{\mathbb{N}}
\newcommand{\Z}{\mathbb{Z}}
\newcommand{\C}{\mathbb{C}}

\begin{document}
\maketitle

\section{Introduction}

\section{Supervised Learning}

We call \textbf{Supervised Learning} to instances of learning where we are given a dataset $(X, Y) \in \R^{(m + n) \times d}$ where $X$ are some inputs and $Y$ are some outputs and each pair $(x, y)$ is an observation or an instance of our dataset. 
The purpose of supervised learning is to learn some mapping that fits the dataset best. 
That is, for some error function $E$, we shall learn a mapping $\hat{f}$ such that $E(\hat{f(X)}, Y)$ is minimized.

\subsection{Linear Regression}

For \textbf{Linear Regression} we assume that the outputs are a linear function of the inputs, and thus our function $\hat{f}$ can be written as $\hat{f}(x) = \theta x$ for some $\theta \in \R^{(m+1)}$.
We define $\theta$ to be in $\R^{(m+1)}$ in order to allow linear functions that do not go through the origin (more precisely, afine functions).
With that in mind, we also extend each input vector $x$ to be in $\R^{(m+1)}$ using the mapping $x \to (1, x)$.

Therefore, we want to find $\hat{\theta} = \arg \min_\theta E(\theta X, Y)$, and our predictions will be of the form $\hat{f}(x) = \hat{\theta}x$.

The most standard way to define $E$ is to use the \textbf{Mean Squared Error} (MSE) defined as 

$$
E(X \theta, Y) = \frac{1}{m} \|X \theta - Y\|^2 = \frac{1}{m} (X \theta - Y)^T (X \theta - Y)
$$

where we assume to be using the Euclidean norm (we explain why bellow).
Equivalently, we can try to minimize the \textbf{Absolute Squared Error} (ASE) that is just the sum of the square differences ($\text{ASE} = m \cdot \text{MSE}$).

Since, we want to minimize this value, we can just differentiate it with respect to $A$ and set it to zero.

\begin{gather*}
\nabla_\theta (E) = 0 \Leftrightarrow 
\nabla_\theta  \left( (X \theta - Y)^T(X \theta - Y) \right) = 0 \Leftrightarrow \\
\nabla_\theta  \left( \theta^T X^T X \theta - \theta^T X^T Y - Y^T X \theta + Y^T Y \right) = 0 \Leftrightarrow \\
2 X^T X \theta - 2 X^T Y = 0 \Leftrightarrow 
\theta = (X^T X)^{-1} X^T Y
\end{gather*}

Note that $X^T X$ is a symmetric matrix, and thus it is invertible if and only if it is positive definite.
This means that we can only use this method if the number of features $m$ is smaller than the number of instances $n$.

The Euclidean norm is the most commonly used norm for measurements in $\R^d$, but that is not why we use it in this instance.
In fact, the Euclidean norm is in some aspects less desirable than other norms (for example, it is very unreliable in outlier detection).
However, it arises naturally by assuming that the errors are normally distributed.
In fact, if we assume that the random variable $X \theta - Y$ is normally distributed with mean zero and variance $\sigma^2$, then the \textbf{Maximum Likelihood Estimation} for $\theta$ is the one that minimizes the MSE.

\end{document}
